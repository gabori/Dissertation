\Chapter{CD-melléklet tartalma}

A szakdolgozatom mellé egy darab CD tartozik, amely a következő adatokat tartalmazza.

\bigskip

\noindent \texttt{Dolgozat} katalógus:

\begin{itemize}
\item \texttt{feladat\_kiiras.pdf}: A feladatkiírást tartalmazó fájl, PDF formátum.
\item \texttt{összefoglalás.pdf}: A dolgozat magyar nyelvű összefoglalása, PDF formátumban.
\item \texttt{összefoglalás.tex}: A dolgozat magyar nyelvű összefoglalása, \LaTeX formátumban.
\item \texttt{summary.pdf}: A dolgozat angol nyelvű összefoglalása, PDF formátumban.
\item \texttt{summary.tex}: A dolgozat angol nyelvű összefoglalása, \LaTeX formátumban.
\end{itemize}

\bigskip

\noindent \texttt{LaTeX} katalógus:

A dolgozatom \LaTeX kódját tartalmazza.

\bigskip

\noindent \texttt{Éttermi rendelés-nyilvántartó\_rendszer.pdf}:

A dolgozatomat tartalmazó PDFfájl.

\bigskip

\noindent \texttt{Forraskód} katalógus:

Az alkalmazásom forráskódját tartalmazza.
