\chapter{Tesztelés}

Miden szoftverben van hiba. A szoftvereket emberek fejlesztik, és az emberek hibáznak, ezért van szükség a programok tesztelésére. A szoftver termékben lévő hibák nagy részét még a végleges üzembe helyezés előtt ki lehet szűrni a tesztek segítségével. A tesztelések segítségével növelhetjük a program megbízhatóságát, minőségét. Teszteléssel se lehet minden hibát kiszűrni, ugyanis nagyobb rendszereknél nem lehetséges minden bemenetet tesztelni. Érdemese a teszteléseket a szoftver életciklusának korai szakaszában elkezdeni, mert minél hamarabb találjuk meg a hibát annál költséghatékonyabb a javítása.

Több fajta tesztelési technikát különböztetünk meg. Beszélhetünk specifikáció alapú tesztről, amikor a tesztelő számára nem áll rendelkezésre a forráskód, és a specifikáció alapján készülnek a tesztek. Illetve létezik strukturális tesztelés, amikor egy kész struktúrát tesztelünk, például osztályokat, funkciókat vagy metódusokat.

\section{Tesztelés szintjei}

A tesztelésnek több szintje is van, léteznek teljes rendszer tesztek,	 amikor azt ellenőrzik, hogy a termék megfelel e a specifikációnak. A rendszerteszt már a kész terméket ellenőrzi. A rendszerteszteket gyakran független cég végzi. Fontos, hogy a tesztkörnyezet minél inkább hasonlítson a megrendelő környezetére.

A tesztelés egy másik szintje az integrációs teszt. Az integrációs teszt az elkészült modulok együttes működését teszteli. Tesztelik a komponensek közti interfészeket, illetve a más rendszerek felé nyújtott interfészeket is.

A tesztelések legalsó a szintje a komponensteszt. A komponensteszt külön-külön teszteli a rendszer egyes komponenseit. Két gyakran használt fajtája van a komponensteszteknek, a modulteszt és a unit-teszt. A modulteszttel általában a nem-funkcionális részeket szokták tesztelni, például, hogy van a memory leak.

\section{Unit-test}

Az egységteszt a funkciókat teszteli. Ismerjük a metódus visszatérési értékét egy adott bemeneti paraméterre, ha a tényleges visszatérési érték egyezik az elvárt értékkel, akkor a teszt sikeres, különben sikertelen. A unit-tesztek lényege, hogy a szoftverben történt változás utána csak újra le kell futtatni a meglévő egységteszteket, és ha egyik teszt eset se bukik el, akkor a változtatás nem okozod hibát a rendszerben.
A unit-tesztek egyik nagy előnye, hogy a legtöbb fejlesztő környezet támogatja őket, tehát egyszerű az implementálásuk. A Phyton esetén is van egy egységtesztelő keretrendszer, amire PyUnit néven szoktak hivatkozni.

A PyUnit a JUnit Python nyelvű változat. Támogatja a tesztek automatizálását, összegyűjtésüket gyűjteménybe, és a tesztek függetlenségét. Az alkalmazásom szerveroldalának tesztelésére unit-teszteket használtam.

\begin{python}
import unittest

class RestaurantTest(unittest.TestCase):
    """Test cases for testing the restaurant class"""
    
    def test_empty_restaurant(self):
        # Set the preconditions ...
        restaurant
        # Check the postconditions ...
        self.assertEqual(x, 10)
\end{python}

