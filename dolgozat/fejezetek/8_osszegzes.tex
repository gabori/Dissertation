\Chapter{Összegzés}

A szakdolgozatom célja egy olyan webalkalmazás készítése volt, amellyel éttermek és rendelések adatait lehet nyilvántartani. Célom volt a felhasználók számára egy egyszerű, letisztult, felhasználóbarát felület kialakítása, amelyen szűrési lehetőségek segítségével könnyen el lehet navigálni az éttermek kínálatai között. Fontos szempont volt az étteremtulajdonosok munkájának megkönnyítése, az éttermek menedzselésének a leegyszerűsítése is. Az adatokat SQL alapú relációs adatbázisban tárolom, amelyhez a keretrendszer SQLAlchemy ORM-en keresztül csatlakozik. Az alkalmazásom megvalósításához szerveroldalon Python/Flask keretrendszert használtam, míg a kliensoldali megvalósítás AngularJS segítségével történt.

Az alkalmazás a jelenlegi állapotában működőképes, a tervezett funkcionalitásokat sikeresen implementáltam. Számos ötlet összegyűlt a dolgozat megírása soran az alkalmazás későbbi továbbfejlesztéséhez. Az egyik ezek közül az ajánlórendszer megvalósítása, amelynek a részletei a dolgozat második fejezetében már kifejtésre kerültek. Egy további ötlet, egy nyersanyag nyilvántartó rendszer bevezetése az alkalmazásba. Az étterem tulajdonosoknak lehetőségük lenne az ételek készítéséhez használt nyersanyag készletük nagyságát felvinni az adatbázisba. Amikor valamelyik nyersanyag készleten lévő darabszáma egy előre meghatározott kritikus érték alá csökken, a rendszer figyelmeztető üzenetet küldene a tulajdonos számára. Hasznos lenne még a rendelési statisztikák és elemzések bővítése is.

\Chapter{Summary}

The goal of my thesis work was to create a web application which manages the data of restaurants and orders. Its aim is to provide a clean, user-friendly interface, which helps the users to filter and navigate among the restaurants. A further plan was to implement functions for helping the work of the restaurant owners, the simplification of restaurant management processes.

The application stores the data in an SQL-based database which accessed via SQLAlchemy ORM. I have used the Python/Flask microframework on the server-side and AngularJS for the frontend application.

I have implemented all of the planned functions of the application. In the further development step, I will add recommendation system features. Its details has mentioned in the chapter 2. An other idea is to introduce a material management subsystem. Therefore, the restaurant owners could manage the data of the amount of raw materials. It makes possible to send an alert messages to the owner when some of the amounts has decreased below its critical value. It could also be useful, to implement more statistics and analysis about the orders.
