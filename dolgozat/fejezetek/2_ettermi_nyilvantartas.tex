\Chapter{Éttermi nyilvántartás}

Az éttermi nyilvántartás egy roppantul összetett problémakör. Az ilyen szoftverek tervezésénél figyelembe kell venni, hogy nincs két egyforma vendéglátó egység, tehát a szoftvert mindig az adott vállalkozáshoz kell igazítani. Nyilván kell tartani a rendeléseket, a raktárkészletet, a felhasználókat, törzsvendégek kezelését. A szoftverrel szemben elvárás, hogy megkönnyítse a szoftverrel dolgozók munkáját, átláthatóvá tegye az üzletvezetők számára az üzlet menetét és az ellenőrzést. Szintén elvárás, hogy könnyen kezelhető és rugalmasan bővíthető legyen.

\Section{Rendelésnyilvántartás}

Rendelések esetén számos adatot nyilván kell tartani. Fontos, hogy ki rendelt, mikor, melyik étteremből, mit, mennyit és mivel fizetett. Menedzselni kell a futárszolgálatot, betartani a kiszállítási időt. Biztosítani kell a vásárlók számára egy pontos leírást az ételekről, választási lehetőséget a fizetési lehetőségek között, estleg vegyes fizetési lehetőségeket egy számlán belül.

\Section{Készletnyilvántartás}

Vannak éttermi nyilvántartó szoftverek, melyek lehetőséget biztosítanak teljes körű raktárkészlet kezelésre, egy vagy akár több raktárral is. Ebbe beletartozik az alapanyagok bevételezése, selejtezés, abban az esetben, ha több raktár is van, akkor raktárok közötti átvételezés.
Ebbe a problémakörbe tartozhat még a menü összeállítása, előrendelések kezelése, a termékekhez tartozó receptúrák kezelése, tehát nyilvántartása és módosítása.

\Section{Törzsvendégek kezelése}

A vásárlók teljeskörű nyilvántartása az ilyen típusú szoftverek esetén alap elvárás. A rendszeresen vásárló felhasználók számára nyújtani kell valamilyen fajta plusz szolgáltatást, amivel meg lehet hálálni a hűségüket, és el lehet azt érni, hogy továbbra is ezt a szoftvert használják. Erre a problémára nyújtanak megoldást a különböző törzsvásárlói kedvezmények.

\Section{Ellenőrzés}

Egy sikeres vállalkozáshoz szükség van a folyamatos ellenőrzésre, az üzletvezetőnek mindig át kell látnia az üzlet menetét. Az ilyen típusú nyilvántartó rendszerek egy része lehetőséget biztosít a termékek szerinti fogyások kimutatására, ez egy nagyon fontos statisztika, amely megmutatja, hogy egy adott időszakban melyik termék volt a legnépszerűbb, tehát fogyott belőle a legtöbb, illetve, hogy melyikből fogyott a legkevesebb. Biztosítják az alapanyagok készletszintjének figyelését.

Fontos statisztikákat, grafikonokat állítanak elő egy adott időszakra, melyek nélkülözhetetlenek a további üzleti lépések meghozatalához, a sikeres üzleti élethez. Kimutatások a pillanatnyi alapanyag készletről, ezeknek lekérdezése, a lekérdezések és kimutatások exportálása.

\Section{Elterjedt megoldások}

Az internetet böngészve számos étterem nyilvántartó szoftvert lehet találni. Vannak köztük online webalkalmazások, és letölthető desktop alkalmazások is. Összeszedtem a már létező alternatívák közül a Magyarországon legelterjedtebbeket \cite{compare_restaurant_softwares}, és egy táblázatban néhány szempont alapján összehasonlítottam őket (\ref{tab:features}. táblázat).

\begin{table}[h!]
\centering
\begin{tabular}{|l|c|c|c|c|c|}
\hline
Program neve & felhasználó- & profil & könyvelés & feladatkezelés & készlet- \\
& kezelés & & & & nyilvántartás \\
\hline
eSystem \cite{esystem} & van & webshop & van & igen & igen \\
\hline
R-keeper & nincs &  egyedi & nincs & igen & igen \\
\hline
MultiStore & nincs &  egyedi & van & igen & igen \\
\hline
Stand-Mágus \cite{standmagus} & van &  egyedi & van & igen & igen \\
\hline
E-étterem & nincs &  egyedi & van & igen & igen \\
\hline
Com-Passz \cite{compassz} & nincs &  egyedi & van & igen & igen \\
\hline
\end{tabular}
\caption{Az elterjedt alkalmazások funkciói}
\label{tab:features}
\end{table}

Nem csak hazánkban ilyen népszerűek az éttermi szoftverek. \Aref{tab:kulfoldi}. táblázatba összeszedtem néhány külföldi megvalósítást \cite{foreign_restaurant_softwares}.

\begin{table}[h!]
\centering
\begin{tabular}{|l|c|c|c|c|c|}
\hline
Program neve & felhasználó- & profil & könyvelés & feladatkezelés & készlet- \\
& kezelés & & & & nyilvántartás \\
\hline
eZee Burrp! & van &  egyedi & van & igen & igen \\
\hline
TapHunter & nincs &  egyedi & nincs & igen & igen \\
\hline
BIM POS & van &  egyedi  & van & igen & igen \\
\hline
Spoonfed & nincs &  egyedi & van & igen & nincs \\
\hline
Aldelo for  & nincs &  egyedi & nincs & igen & igen \\
Restaurants & & & & & \\
\hline
\end{tabular}
\caption{Néhány elterjedt, külföldi alkalmazás}
\label{tab:kulfoldi}
\end{table}

A táblázat első oszlopában a szoftver neve szerepel. A második oszlopban a felhasználókezelés, ami azt takarja, hogy a szoftver funkció között van-e vendég nyilvántartás, tehát, hogy lehet-e regisztrálni a vendégeket. A harmadik oszlopban a szoftver profilját, arculatát jellemeztem. A negyedik oszlopban azt vizsgáltam, hogy van-e integrált könyvelés funkció az alkalmazásban. Az 5. oszlopban feladatkezelés alatt azt értettem, hogy a pénztárgép funkcióit is a szoftver látja-e el. A hatodik azaz utolsó oszlopban a szoftver raktár készlet kezelését vizsgáltam. Az én alkalmazásom a vásárlók teljes körű nyilvántartására, rendelések nyilvántartására és az ezekből alkotott kimutatásokra, statisztikákra fog összpontosítani. Az alkalmazásom célja, a házhoz szállítást nyújtó éttermek összegyűjtése, az üzletvezetők feladatának, éttermek menedzselésének megkönnyítése, és a rendelők számára a minél egyszerűbb ételrendelés biztosítása. Az hasonló célú és felépítésű szoftvereket online ételrendelési portálnak szokták nevezni.

A magyar piacon jelenleg három online ételrendelési portál óriás van, a NetPincér \cite{netpincer.hu}, a Pizza.hu \cite{pizza.hu} és a Falatozz.hu \cite{falatozz.hu}. Ezek a portálok közvetítő szerepet játszanak a megrendelő és a kiszállítást lebonyolító étterem között. Az éttermeknek ez egy nagyon jó marketing fogás, segít nekik az ügyfélkör kibővítésében, a megrendelőknek pedig leegyszerűsítik az online ételrendelést, azzal, hogy egy oldalon megtalálják a környéken lévő összes vendéglátó egység kínálatát.

Az alkalmazásom abban különbözik ezektől az online étrendelési portáloktól, hogy míg ezek a portálok úgymond csak közvetítő szerepet vállalnak, nem feltétlenül tartják nyilván az adatokat, a partner cégek egyébként is rendelkeznek egy saját étterem nyilvántartói rendszerrel és általában webes rendelő felülettel is. Addig az általam készített alkalmazás biztosítja a rendelésnyilvántartást, a vásárlónyilvántartást és a szerzett adatokból olyan statisztikákat, kimutatásokat állít elő, melyek elengedhetetlenek egy sikeres vállalkozáshoz.

\Section{Ajánlórendszer}

Ajánlórendszerekkel nap mint nap találkozunk az interneten \cite{recommender_system}. Az ajánlórendszereknek köszönhetjük az online vásárlás közben nekünk ajánlott termékeket, videó nézés közben a videókat, melyek számunkra érdekesek lehetnek, vagy éppen böngészés közben olyan cikkek jelennek meg, melyek felkelthetik érdeklődésünket. Ezek rendszerek lényegében speciális információ szűrű rendszerek. Szükségük van a felhasználók adatbázisára, a termékek adatbázisára és a köztük lévő kapcsolatra. Az előbbi adatokból profilokat építenek tanuló algoritmusok segítségével. A kapott modellek alapján a rendszer már képes olyan tartalmat ajánlani, ami érdekes lehet a felhasználó számára.

\SubSection{Az ajánlási feladat definíciója}

Ahogy már fentebb említettem, szükség van a felhasználók profiljára. A felhasználók jelölése: $U$. A termékek jelölélese: $I$. Adott egy $f: U \times I \rightarrow R$ leképzés, ahol $R$ egy halmaz teljes rendezéssel. A cél meghatározni az $\hat{f}: U \times I \rightarrow R$ leképzést, amely teljesen definiált az $U \times I$ téren, és a lehető legjobban közelíti $f$-et.

\SubSection{Az ajánlási feladat megoldása}

Az ajánlási feladat megoldása egy olyan algoritmus, ami meghatározza $\hat{f}$-et. Az ajánlórendszer az $\hat{f}$-et fogja használni az ajánlatok tételénél.

\SubSection{Algoritmusok osztályozása}

Ezeket az algoritmusokat többféle szempont szerint osztályozhatjuk. Létezik osztályzás az algoritmusban használt megközelítés alapján, ide tartoznak a gráf alapú módszerek, a gépi tanulás modellt használó eljárások, és például a hasonlóságon alapuló módszerek.

Lehet osztályozni felhasznált információk alapján is. Ebbe a csoportba tartozik a content based methods, amikor a felhasználó korábbi értékelései alapján történik az ajánlattétel, illetve a collaborative filtering, amelyben a kapott inputhoz hasonló adatok is felhasználásra kerülnek az ajánlattétel során. Ennek két típusa van, a user-based, amikor a cél felhasználó ízléséhez hasonló ízlésű felhasználók szavazatait is figyelembe veszik az ajánlattételkor. A másik típus az item-based algoritmusok, amikor a választott termékhez hasonló termékeket ajánlja fel a rendszer.

\newpage

\SubSection{Ontológia}

Az ontológiák használata megoldást biztosít az alkalmazások számára, hogy fel tudják kutatni az adatbázisok belső jelentéseit \cite{owl}. Az adatbázisokban az egyes fogalmak más-más jelentéssel bírhatnak. Egy olyan programnak, amely információt használ fel ezekből az adatbázisokból, tudnia kell, hogy az általa alkalmazott fogalmak ugyanazt a dolgot jelentik.

Az ontológia a fogalomalkotás vagy fogalomfeltérképezés specifikációját, azaz a körülírását, megkülönböztetését jelenti. Az ontológia egy formális és explicit leírása egy elosztott elképzelésnek. 

A mesterséges intelligencia területén ontológiákat a tudás megosztása és újbóli felhasználása céljából hoznak létre. A tudás átadása úgy történik, hogy különböző tulajdonságokat rendelünk az egyes osztályokhoz, és lehetővé tesszük, hogy a leszármazottjaik örököljék ezeket a tulajdonságokat.

Az ontológiák egyik legismertebb változata a pizza ontológia, amit a Manchesteri Egyetemen fejlesztettek ki tanulmányi célokból. Ez egy egyszerű, könnyen értelmezhető ontóliga. Nagy mennyiségű relációt ki lehet fejezni a pizzákhoz rendelt feltét tulajdonságok alapján.

A pizza ontológia segítségével meg lehetne valósítani egy ajánlórendszert az alkalmazásomban. A megvalósításhoz az adatbázis minimális bővítésére lenne szükség, a rendelésekből kinyert adatok, felhasználói szokások segítségével könnyen és hatékonyan meg lehetne valósítani.

Az ajánlórendszer működése a pizza ontológián alapulna, amit OWL segítségével tudnék definiálni és példányosítani. Az OWL, azaz Web Ontology Language, megkönnyíti a webes tartalmak értelmezését kiegészítő szókincs használatával és formális jelentéstannal. Minden egyes pizza egy ős pizzából származik, a feltétek széles körében lehet válogatni. Minél több azonos feltét van két különböző pizzán, annál nagyobb valószínűséggel fogja őket ajánlani a felhasználóknak. Amennyiben a felhasználó olyan pizzákat rendel, melyeken hús alapú feltét is szerepel, a rendszer nem fog vegetáriánus pizzát ajánlani neki.

A fentiekben leírt ajánlórendszer megvalósítása lesz az első lépés az alkalmazásom továbbfejlesztésében. 

