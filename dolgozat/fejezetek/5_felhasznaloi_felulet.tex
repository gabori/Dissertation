\Chapter{A felhasználói felület}

A fejezet a felhasználói felület működését, megjelenítési módját és használatát mutatja be. A következő szakaszok kategóriákra bontva részletezi ezeket.

\Section{Bejelentkezés/regisztráció}

Az alkalmazás használatához érvényes felhasználói fiókra van szükség. Az alkalmazást használó felhasználókat két típusra lehet bontani, vannak az egyszerű userek(vásárlók), akik ételeket tudnak rendelni az adatbázisban szereplő éttermekből, és vannak az étteremtulajdonosok, akik saját éttermüket tudják menedzselni. Az alkalmazás további szolgáltatásainak az igénybevételéhez mindenképp be kell jelentkezni vagy userként, vagy pedig étteremtulajdonosként. Amennyibben a felhasználó most először látogatott el az oldalra, és még nincs felhasználói fiókja, van lehetősége userként regisztrálni magát. Ehhez mindössze egy egyszerű űrlapot kell kitöltenie, ahol többek között meg kell adnia a nevét, elérhetőségét, címét. Abban az esetben, ha valaki étteremtulajdonosként szeretne regisztrálni, meg kell keresni egy e-mail-el az oldal üzemeltetőjét.

\begin{figure}
\centering
\includegraphics[scale=0.8]{kepek/login.png}
\caption{Bejelentkező felület}
\label{fig:architecture}
\end{figure}

\Section{Egyszerű felhasználói funkciók}

A továbbiakban az alapfunkciókat aszerint fogom leírni, hogy melyik felhasználói csoport számára érhetők el. A vásárlók által elérhető szolgáltatásokkal kezdem.

\SubSection{Éttermek listázása}

A vásárlók számára adott egy olyan funkció, hogy ki tudják listázni az adatbázisban szereplő összes éttermet. Az éttermek egymás alatt, egy táblázatban jelennek meg. A táblázatban szerepel az étterem logója, neve, rövid leírása, címe, a várható kiszállítási idő, kiszállítási költség, a minimális rendelés összege és egy gomb, amire kattintva az oldal tovább irányítja a vásárlót az adott étterem étlapjához (\ref{fig:restaurants}. ábra).

A vásárlónak lehetősége van szűrni az éttermeket város szerint, ami segíti őket abban, hogy csak a számukra elérhető távolságban levő éttermek kínálatát lássák.

\begin{figure}
\centering
\includegraphics[scale=0.48]{kepek/restaurants.png}
\caption{Az éttermek listázása}
\label{fig:restaurants}
\end{figure}

\SubSection{Étlap}

Miután a vásárló kiválasztotta a neki legszimpatikusabb éttermet, az „innen rendelek” gombra kattintva el lehet érni az adott étterem étlapját.

Az ételek egymás alatt, egy táblázatban jelennek meg (\ref{fig:menu}. ábra). A táblázat egyes soraiban egy-egy étel szerepel, az oszlopaiban pedig az adott étel adatai, többek között neve, egy kép róla, az ára és egy rövid leírás róla. A táblázat jobb szélső oszlopában egy „kosárba” feliratú gomb található, amit megnyomva a kiválasztott termék belekerül a felhasználó kosarába.

Az étlap felett van egy legördülő menü, ahol vásárló kiválaszthatja, hogy milyen típusú ételek között szeretne válogatni, milyen típusút szeretne rendelni. Az ételtípus kiválasztása után, az ételek listája szűrve lesz az adott típus szerint, tehát csak a kívánt típusú ételek fognak megjelenni a táblázatban.

Miután a vásárló kiválasztotta a rendelni kívánt termékeket, és hozzáadta őket a kosárhoz, megjelenik a felületen a „fizetés” gomb, amire kattintva elérjük a fizetés funkciót.

\begin{figure}
\centering
\includegraphics[scale=0.65]{kepek/menu.png}
\caption{Kép egy, az alkalmazásban megjelenített étlapról}
\label{fig:menu}
\end{figure}

\SubSection{Fizetés}

A felhasználói felületen a kosár az étlap mellett, jobb oldalon található szintén egy táblázatos megoldással. A hozzáadott termékek neve és ára egymás alatt jelenik meg, egy végösszeggel a táblázat alján (\ref{fig:order}. ábra).

A kosár legalsó sorában egy „fizetés” gomb található. A gombra kattintva felugrik egy modal ablak, ahol egy legördülő menüből lehet kiválasztani a kívánt fizetési módot. Az alkalmazásban korlátozott számú fizetési lehetőség közül lehet választani, ami éttermenként változó lehet. Hogy egy étterem milyen fizetési lehetőségeket biztosít, azt az adott étterem felvitelekor kell megadni.

\begin{figure}
\centering
\includegraphics[scale=1]{kepek/order.png}
\caption{A vásárlás tételei összeggel és végösszeggel megjelenítve}
\label{fig:order}
\end{figure}

A kívánt fizetési lehetőség kiválasztása után (\ref{fig:payment}. ábra) a fizetés gombbal lehet véglegesíteni a megrendelést. Ilyenkor kapunk értesítést, hogy sikeres volt a rendelés.

\begin{figure}
\centering
\includegraphics[scale=0.8]{kepek/payment.png}
\caption{A fizetési mód kiválasztása}
\label{fig:payment}
\end{figure}

\SubSection{Profilom}

A profilom funkció kilistázza a bejelentkezett felhasználó adatait. Egy táblázatba foglalva kilistázza a felhasználó teljes nevét, felhasználónevét, e-mail címét, telefonszámát, jutalompontját és címét/címeit. Egy felhasználónak akár több különböző szállítása címe is lehet. A táblázat legalsó sorában van egy gomb „jelszó módosítása” felirattal (\ref{fig:profile}. ábra).

Jutalompontot a rendelések után kap a vásárló, minden elköltött 1000 forint után három jutalom pontot ír jóvá a rendszer. Az összegyűjtött pontokat, fizetéskor be lehet váltani, ilyenkor a beváltott pontok összege levonásra kerül a rendelés végösszegéből.

\begin{figure}
\centering
\includegraphics[scale=0.6]{kepek/profile.png}
\caption{Felhasználói adatok megjelenítése}
\label{fig:profile}
\end{figure}

A „Jelszó módosítása” gombra kattintva megjelenik egy űrlap, ahol a felhasználónak lehetősége van megváltoztatni a jelenlegi jelszavát (\ref{fig:password}. ábra). Meg kell adnia a jelenlegi jelszavát, majd az új jelszavát és végül meg kell erősítenie az új jelszót. Ha kitöltötte a bemeneti mezőket, a „Mentés” gombra kattintva megtörténik a jelszó cserélő funkció meghívása. Amennyibben a jelenlegi jelszó helyes, az új jelszó megfelel a kritériumoknak és megegyezik a megerősített jelszóval, megtörténik a jelszó módosítása.

\begin{figure}
\centering
\includegraphics[scale=0.8]{kepek/password.png}
\caption{Jelszó megváltoztatása}
\label{fig:password}
\end{figure}

\Section{Étterem tulajdonos funkciók}

\SubSection{Éttermeim}

Az éttermeim funkció kilistázza a felhasználó adatbázisban szereplő éttermeit. Az éttermek táblázatos elrendezésben jelennek meg a felületen. A táblázat minden egyes sora, egy éttermet reprezentál. A táblázat oszlopaiban az éttermek különböző adatai találhatók, többek között az étterem logója, neve, címe. A jobb szélső oszlopban négy gomb található, „Étlap szerkesztése”, „Étterem szerkesztése”, „Rendelések” és a „Statisztikák” (\ref{fig:my_restaurants}. ábra).

\begin{figure}
\centering
\includegraphics[scale=0.5]{kepek/my_restaurants.png}
\caption{A saját éttermek kilistázása}
\label{fig:my_restaurants}
\end{figure}

\SubSection{Étlap szerkesztése}

Az étlap szerkesztése menüpont kiválasztása után, egy táblázatot fogunk látni az adott étterem által forgalmazott ételekről és ezen ételek adatairól. Az elrendezés hasonló a user oldalon látott étlapok elrendezéséhez.

Az ételek egymás alatt, egy táblázatban jelennek meg. A táblázat egyes soraiban egy-egy étel szerepel, az oszlopaiban pedig az adott étel adatai. A táblázat jobb szélső oszlopában két gomb található, „Étel szerkesztése”, „Étel törlése”. A táblázat alatt egy „Étel felvitele” feliratú gomb van.

A táblázat felett egy legördülő menü van, amelynek segítségével a tulajdonos rá tud szűrni a különböző ételtípusokra, ezáltal könnyebben megtalálhatja a szerkeszteni, vagy törölni kívánt terméket. Az ételtípus kiválasztása után, az ételek listája szűrve lesz az adott típus szerint, tehát csak a kívánt típusú ételek fognak megjelenni a táblázatban (\ref{fig:new_meal}. ábra).

\begin{figure}
\centering
\includegraphics[scale=0.6]{kepek/new_meal.png}
\caption{Új étel hozzáadása}
\label{fig:new_meal}
\end{figure}

\SubSection{Étel törlése}

Az étteremtulajdonosnak lehetősége van testre szabni az étterme által biztosított kínálatot. Vannak bizonyos ételek, melyeket csak szezonálisan lehet elkészíteni az alap\-anyag-szükségletük miatt, vagy csak valamilyen oknál fogva úgy dönt a vezetőség, hogy le kell kerülnie az étlapról. Ilyenkor van szükség az étel törlése funkcióra. Az „Étel törlése” gombra kattintva meghívásra kerül az étel törlése funkció, melynek hatására a kiválasztott étel törlődik a megjelenített ételek listájából, és törlődik az adatbázisból is.

\SubSection{Étel szerkesztése}

Az étel szerkesztése funkcióval lehetőséget biztosítok az étteremtulajdonos számára, az étlapon szereplő termékek adatainak módosítására. Tehát ha egy adott ételnek megváltozik az ára vagy például az összetétele, akkor nem kell törölni azt, majd az új adatokkal ismét felvinni az adatbázisba. Mindössze rá kell kattintani a szerkeszteni kívánt termék sorában található gombok közül az „Étel szerkesztése” gombra, melynek hatására meghívódik az étel szerkesztése funkció.

Egy űrlapot fogunk látni, ahol a szükséges módosítások elvégzése után, a mentés gombbal lehet véglegesíteni a módosítást (\ref{fig:edit_meal}. ábra).

\begin{figure}
\centering
\includegraphics[scale=0.8]{kepek/edit_meal.png}
\caption{Egy étel adatainak szerkesztése}
\label{fig:edit_meal}
\end{figure}

\SubSection{Étel felvitele}

Az étel felvitel gombra kattintva egy űrlap fog betöltődni. Az űrlap beviteli mezőkből áll, melyekkel az étlapra felvinni kívánt étel adatait tudjuk elküldeni a szervernek. Meg kell adni többek között az étel nevét, árát és típusát. Amennyiben minden szükséges mezőt kitöltöttünk a „Felvitel” gombra kattintva lehet véglegesíteni a műveletet.

\begin{figure}
\centering
\includegraphics[scale=0.8]{kepek/add_meal.png}
\caption{Új étel felvitele az étlapra}
\label{fig:add_meal}
\end{figure}

\SubSection{Rendelések}

Ahogy már említettem, az éttermeim funkció kilistázza a felhasználó adatbázisban szereplő éttermeit. Ilyenkor az éttermek táblázatos formában jelennek meg, minden étterem sorában szerepel egy „Rendelések” feliratú gomb. Erre a gombra kattintva a szerver kilistázza az adott étteremhez tartozó megrendeléseket, tehát azokat a rendeléseket, amiket ebben az étteremben adtak le a vásárlók. A rendelések táblázatos formában jelennek meg a felületen, egy sor egy rendelés. A táblázat oszlopaiban az egyes rendeléshez tartozó rendelési adatok vannak megjelenítve. Az adatok között szerepel a rendelő neve, rendelés időpontja, a megrendelt termékek, a rendelés összege és a fizetési módja.

\SubSection{Étterem szerkesztése}

Egy étteremtulajdonosnak lehetősége lesz az éttermei adatainak módosítására. A szerkesztendő étterem sorában rá kell klikkelni az étterem szerkesztése gombra. Kattintás után megjelenik egy űrlap, amire fel kell vinni a módosítandó adatokat. Az adatok megadását követően a mentés gombbal lehet véglegesíteni a módosításokat (\ref{fig:edit_meal}. ábra).

\begin{figure}
\centering
\includegraphics[scale=0.9]{kepek/edit_restaurant.png}
\caption{Étterem adatainak szerkesztése}
\label{fig:edit_restaurnt}
\end{figure}

\SubSection{Étterem hozzáadása}

Egy étteremtulajdonosnak több étterme is lehet az adatbázisban. Az étterem hozzáadása menüpont alatt van lehetőség egy újabb éttermet felvinni. 
A menüpontra kattintva egy űrlapot fogunk látni, ahol a beviteli mezők egymás alatt helyezkednek el. Az egyes beviteli mezőkben az étterem különböző adatait lehet megadni. Az étterem nevét, leírását, címét. Ha elkészültünk az adatok megadásával, a „Felvitel” gombbal lehet véglegesíteni ezt a műveletet (\ref{fig:add_restaurant}. ábra).

\begin{figure}
\centering
\includegraphics[scale=0.9]{kepek/add_restaurant.png}
\caption{Étterem hozzáadása}
\label{fig:add_restaurant}
\end{figure}

\SubSection{Profilom}

A profilom funkció az étteremtulajdonosok szempontjából is ugyanaz, mint a vásárlóknál. Kilistázza a bejelentkezett felhasználó adatait. Egy táblázatba foglalva kilistázza a felhasználó teljes nevét, felhasználónevét, e-mail címét, telefonszámát és címét. A táblázat legalsó sorában van egy gomb „jelszó módosítása” felirattal.

A jelszó módosítás is ugyanúgy működik, mint a vásárlóknál. A „Jelszó módosítása” gombra kattintva megjelenik egy űrlap. Meg kell adni a jelenlegi jelszót, majd az új jelszót és végül meg kell erősíteni az új jelszót. A „Mentés” gombra kattintva megtörténik a jelszó cserélő funkció meghívása. Amennyibben a jelenlegi jelszó helyes, az új jelszó megfelel a kritériumoknak és megegyezik a megerősített jelszóval, megtörténik a jelszó módosítása.

\Section{Felhasználói felület megvalósítása AngularJS segítségével}

A felhasználói felületet AngularJS segítségével valósítottam meg. Az app.js fájl tartalmazza az Angular kódot. Az alkalmazásom egy single-page application, ami azt jelenti, hogy a weblap folyamatosan kis mennyiségű adatot cserél a szerverrel, tehát nem teljes HTML oldalak töltődnek le a szerverről, hanem egy adott oldal tartalma dinamikusan változik az alkalmazás használata során. A felhasználói felület különböző nézetekből épül fel. A nézetek HTML fájlokban tárolt HTML-kódrészletek. Egy nézet létrehozásához implementálni kell egy kontrollert és definiálni egy állapotot. A kontrollerek feladata a szerverrel való kommunikáció, és az adatok megjelenítése a nézeteken. A \texttt{\$stateProvider} modul kezeli az állapotokat, definiálásuk a \texttt{.state()} metódussal történik. Egy állapot definiálásakor meg kell adni a nevét, az URL-t amire aktiválódik, a HTML fájl elérési útját, amit be kell töltenie, tehát a nézetet és azt, hogy melyik kontrollert használja. Az \texttt{\$urlRouterProvider} modul \texttt{.otherwise()} metódusa segítségével létrehoztam egy olyan állapotot, ami akkor aktiválódik, ha a hivatkozott URL egyik meglévő állapotra se illeszkedik. Ilyenkor egy hiba üzenetet tartalmazó nézet töltődik be.

Az angular fájlban definiálok egy \texttt{.run} blokkot is, amelyben a sütik értékeit frissítem, és biztosítom, hogy amíg a felhasználó nem jelentkezett be, csak a bejelentkező és regisztráló felületet érhesse el.

A fejezet további részében az állapotokat és kontrollereket fogom részletezni.

\SubSection{Állapotok}

Ahogy már korábban leírtam egy állapot definiálásakor meg kell adni a nevét, az URL-t amire aktiválódik, a HTML fájl elérési útját, amit be kell töltenie, tehát a nézetet, és azt, hogy melyik kontrollert használja. A következő képpel demonstrálom ezt:

\begin{cpp}
.state("/", {
    url: "",
    templateUrl: "/static/partials/home.html",
    controller: "loginController"
})
\end{cpp}

A fenti állapot a \texttt{/static/partials/home.html} nézetet tölti be a \texttt{login\-Control\-ler} segítségével, ami a bejelentkező felület. A \texttt{loginController} a felhasználók be és ki jelentkezését, illetve az autentikációt valósítja meg. Az autentikáció megvalósításához létrehoztam egy \textit{service}-t, az \texttt{AuthenticationService}-t. Ezt a \texttt{service.js} fájl valósítja meg. A hitelesítési folyamat során a kontroller meghívja az \texttt{AuthenticationService} login metódusát, és paraméterként átadja neki a felhasználó által megadott felhasználónevet és jelszót. A service a kapott adatokat elküldi a szervernek, és várja a választ. A szerver elvégzi a szükséges ellenőrzéseket, lekérdezéseket, token generálást és választ küld a \textit{service}-nek. A \textit{service} a kapott választ továbbítja a kontroller felé. A kontroller megvizsgálja a választ, ha nem 200-as státuszkódot kap, akkor visszatér egy hibaüzenettel. Ha 200-as státuszkódot kapott a hitelesítés sikeres volt, meghívja az \texttt{AuthenticationService} \texttt{SetCredentials} metódusát és át adja neki a felhasználó nevét, jelszavát és típusát, majd betölti a \texttt{/startpage} nézetet. A service \texttt{SetCredentials} függvénye egy \texttt{\$rootScope} változónak dictionary-ként értékül adja a kapott adatokat és a szerver által generált tokent, majd értékül adja a változót egy sütinek.

Kijelentkezés esetén a kontroller az \texttt{AuthenticationService} \texttt{ClearCredentials()} metódusát hívja meg, ami törli a bejelentkezéskor létrehozott sütit.

\bigskip

\noindent \texttt{/startpage}

A \texttt{/startpage} URL-re hivatkozva a \texttt{/static/partials/startpage.html} nézet töltődik be, ami a webalkalmazás nyitóoldala.

\bigskip

\noindent \texttt{/registration}

A \texttt{/registration} URL-re való hivatkozás a \texttt{/static/partials/} \\ \texttt{registration.html} nézetet tölti be a \texttt{userController} segítségével. A kontroller betölti a regisztrációs felületet, a felhasználó felviszi az adatait, amiket a kontroller továbbít a szerver felé.

\bigskip

\noindent \texttt{/restaurant}

A \texttt{/restaurant} URL-re hivatkozva a \texttt{restaurantController} betölti a \\ \texttt{/static/partials/restaurant.html} nézetet. A kontroller lekéri a szervertől az éttermeket. A megjelenített táblázatban biztosítja a városok szerinti szűrést.

\bigskip

\noindent \texttt{/myRestaurant}

A \texttt{myRestaurantController} a \texttt{/static/partials/my\_restaurants.html} nézetett tölti be a \texttt{/myRestaurant/:username} URL hivatkozás után. A kontroller lekéri a szervertől a paraméterként átadott azonosítóval rendelkező felhasználó éttermeit.

\bigskip

\noindent \texttt{/list-meals}

A \texttt{/list-meals/:restaurant\_id} URL-re hivatkozva a \texttt{mealsController} betölti a \texttt{/static/partials/list\_meals.html} nézetet. A kontroller lekéri a szervertől a paraméterként kapott étterem által kínált ételeket, az étterem által kínált ételek típusait és a fizetési módokat. A táblázatban megjelenített ételek között a kontroller szűrési lehetőséget biztosít a lekért típusok alapján. A kontroller létrehoz egy \texttt{\$rootScope} kosarat, amihez az \texttt{addToCart()} funkció meghívásával lehet ételeket hozzáadni. Az \texttt{addToCart()} kap egy ételt paraméterként, amit beletesz a kosár listába, majd a kosarat beleteszi egy sütibe. A \texttt{removeFromCart()} függvénnyel lehet a kosárból törölni, először a kosár listából törli a paraméterként kapott ételt, majd frissíti a süti tartalmát.

A kontrollerben definiálva van egy \texttt{checkout()} függvény, ami fizetéskor kerül meghívásra. Fizetés előtt ki kell választani a szervertől lekérdezett fizetési módok közül egyet. A fizetés-re kattintva meghívásra kerül a fentebb említett \texttt{checkout()} metódus, ami elküldi a szervernek a kosár tartalmát, a vásárló nevét, és a választott fizetési módot.

\bigskip

\noindent \texttt{/setting-meals}

A \texttt{/setting-meals/:restaurant\_id} URL-re hivatkozva a \texttt{meals\-Setting\-Control\-ler} betölti a \texttt{/static/partials/setting\_meals.html} nézetet. A kontroller lekéri a szervertől a paraméterként megadott étterem ételkínálatát, és az étterem által forgalmazott ételek típusait. A lekérdezett ételeket egy táblázatban jeleníti meg, ahol ételtípus szerinti szűrési lehetőséget is biztosít.

A kontrollerben definiálva van egy \texttt{removeMeal()} függvény, ami a paraméterként kapott étel objektumot elküldi a szervernek, a szerver törli az adatbázisból, majd visszatér egy státuszkóddal.

\bigskip

\noindent \texttt{/add-meal}

Az \texttt{/add-meal} állapot a \texttt{/add-meal/:restaurant\_id} URL-re hivatkozva érhető el. A hivatkozás során a \texttt{mealsAddController} betölti a \texttt{/static/partials/} \\ \texttt{add\_meal.html} nézetet. A kontrollernek egyetlen \texttt{addMeal()} metódusa van, ami az új étel felviteléért felelős. A metódus kap egy étel objektumot, ami a felhasználó által megadott adatokat tartalmazza. A szervernek paraméterként elküldi a kapott objektumot, és a módosítandó étterem azonosítóját. A szerver elvégzi a szükséges adatbázismódosításokat, majd visszatér egy státuszkóddal.

\bigskip

\noindent \texttt{/addRestaurant}

Az \texttt{/addRestaurant} URL-en a \texttt{/static/partials/add\_restaurant.html} nézet érhető el, melynek megjelenítéséért az \texttt{addRestaurantController} a felelős. A kontrollerben van egy \texttt{addRestaurant()} függvény, amit meghívva elküldi a szervernek a paraméterként kapott adatokat, és a bejelentkezett felhasználó nevét. A szerver elvégzi a módosításokat, felviszi az adatbázisba az új éttermet, majd sikeres adatbázisműveletek esetén 200-as státuszkódot küld vissza válaszként.

\bigskip

\noindent \texttt{/my-orders}

A \texttt{/static/partials/get\_orders.html} nézet a \texttt{/my-orders/:restaurant\_id} \\ URL-en érhető el. A nézet megjelenítését az \texttt{ordersController} végzi. A kontroller lekéri a szervertől a paraméterként kapott étteremben leadott összes rendelést, majd egy táblázatban megjeleníti azokat.

\bigskip

\noindent \texttt{/my-profile}

A \texttt{/my-profile} URL-re hivatkozva elérhetjük a \texttt{/static/partials/} \\ \texttt{my\_profile.html} nézetet, melyet a \texttt{userProfileController} jelenít meg. A kontroller lekéri a szervertől az aktuális bejelentkezve lévő felhasználó felhasználói adatait.

\bigskip

\noindent \texttt{/edit-profile}

Az \texttt{/edit-profile/:user\_id} URL-re való hivatkozás után a \texttt{userSettingsCont\-rol\-ler} megjeleníti a \texttt{/static/partials/edit\_password.html} nézetet. A kontroller lekéri a szervertől az aktuális bejelentkezve lévő felhasználó felhasználói adatait. A kontrollerben definiálva van egy \texttt{editUser()} metódus, ami paraméterként elküldi a szervernek a lekért felhasználói adatokat, és a felhasználói felületről kapott adatokat. A szerver feldolgozza az adatokat, elvégzi az ellenőrzéseket, és módosítja a felhasználó jelszavát. Sikeres módosítás esetén 200-as státuszkódot küld vissza válaszként.

\bigskip

\noindent \texttt{/edit-restaurant}

Az \texttt{editRestaurantController} betölti a \texttt{/static/partials/} \\ \texttt{edit\_restaurant.html} nézetet az \texttt{/edit-restaurant/:restaurant\_id} URL-re hivatkozás után. A kontroller lekéri a szervertől a paraméterként kapott étterem adatait. A kontrollerben definiált \texttt{editRestaurant()} függvény paraméterként elküldi a szervernek a lekérdezett étterem adatait, és a felhasználói felületről kapott módosítandó adatokat. A szerver elvégzi a szükséges ellenőrzéseket és adatbázismódosításokat, majd visszatér egy státuszkóddal.

\bigskip

\noindent \texttt{/edit-meal}

A \texttt{/static/partials/edit\_meal.html} nézetet a \texttt{/edit-meal/:meal\_id} URL-en lehet elérni. A nézetet az \texttt{editMealController} jeleníti meg. A kontroller lekéri a szervertől a paraméterként kapott étel adatait. A kontrollerben definiált \texttt{editMeal()} metódus paraméterként elküldi a szervernek a lekérdezett étel adatait, az ételt forgalmazó étterem azonosítóját, és a felhasználói felületről kapott módosítandó adatokat. A szerver elvégzi a szükséges ellenőrzéseket és adatbázismódosításokat, majd visszatér egy státuszkóddal.
