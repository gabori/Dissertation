\chapter{Alapfunkciók bemutatása}

Ez nem feltétlen jó így fejezet címnek, pontosabban ezen a témán belül több fejezet is lehet, attól függően, hogy majd mivel és hogy sikerül elkészülni.

Itt kb. olyasmire kell gondolni, mint
- rendelések nyilvántartása,
- promóciók, törzsvásárlói kedvezmények megvalósítása a rendszerben,
- nyersanyagok nyilvántartása,
- rendelési statisztikák, elemzés, megjelenítés,
- ...

Annyi biztos, hogy a címnek megfelelően a rendeléseket majd nyilván kell tartani. A többi ahhoz kapcsolható funkció, de csak akkor célszerű vele foglalkozni, ha az alapfunkciók már megvannak.

A tervezéséről szóló dolgokat az alkalmazás felépítésénél is el lehet kezdeni, de ott még inkább csak az interfészekre, az API jellegére, konvenciókra, illesztési módokra vonatkozóan.

Magában az implementációba annyira kell csak majd belemenni az alapfunkciók bemutatásánál, hogy a kiemelt kódpéldák alapján át lehessen tekinteni a rendszer egészét, és a leírás alapján aki akarja hellyel-közzel tuda is reprodukálni a rendszert.
