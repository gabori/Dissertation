\Chapter{Bevezetés}

A világ felgyorsult, az emberek állandó rohanásban vannak, nem mindig jut idő olyan alapvető dolgok elvégzésére, mint például a főzés. Az éttermek látogatása is egy idő igényes tevékenység, ezért megnőtt a kereslet az ételek házhoz rendelésére. Az interneten számos alternatívát találhatunk ételrendelős webes alkalmazások terén, de ezek nem működnek megfelelően, nem egyértelmű a használatuk és nem felhasználóbarát az alkalmazások kialakítása. A dolgozatom célja egy olyan webalkalmazás készítése, amely több étterem adatbázisát kezeli, ezáltal megkönnyíti az ételrendelést a felhasználók számára, mivel egy weboldalon megtalálhatják számos étterem kínáltatát, és kiválaszthatják a számukra legszimpatikusabbat. A dolgozatomban összpontosítani fogok egy letisztult, egyértelmű, könnyen kezelhető felhasználói felület kialakítására különböző szűrési lehetőségekkel. Továbbá az üzletvezetők feladatának, az éttermek menedzselésének megkönnyítésére online étterem kezelő felülettel, statisztikákkal, kimutatásokkal, melyek elengedhetetlenek egy sikeres vállalkozáshoz.

Az alkalmazásom szerveroldalon Python/Flask keretrendszert, míg kliensoldalon AngularJS-t fog használni. Az adatokat relációs adatbázis tárolja majd, amelyhez a keretrendszer SQLAlchemy ORM-en keresztül fog csatlakozni.

A dolgozatom első részében, el fogom helyezni az alkalmazásomat a piacon, bemutatom az étterem és a rendelésnyilvántartás módjait, és néhány konkurens programot. Azután szeretném bemutatni a használni kívánt technológiákat, és ismertetem az alkalmazásom felépítését, főbb részeit. Ezt követően részletezem az SQL adatbázisom felépítést, majd leírom az alkalmazás kliens és szerveroldali megvalósítását. És végül kitérek az elkészített alkalmazásom tesztelésének lehetőségeire is.
